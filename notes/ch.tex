% $Id: ch.tex,v 1.18 2006/04/03 21:13:28 bolo Exp $

\documentclass[a4paper,11pt]{article}
\usepackage[latin2]{inputenc}
\usepackage{a4wide}
%\usepackage[OT4]{polski}
%\usepackage[polish]{babel}
\usepackage[english]{babel}
\usepackage{amsfonts}
\usepackage{mathrsfs}
\usepackage{alltt}
%\usepackage{sans}

\bibliographystyle{siam}


\title{Notes on the Cahn-Hilliard equation}
\date{}
\author{�ukasz Bolikowski (bolo@icm.edu.pl)}

\begin{document}

%\thispagestyle{empty}
\maketitle

\begin{alltt}
{\scriptsize Version: $Id: ch.tex,v 1.18 2006/04/03 21:13:28 bolo Exp $}
\end{alltt}

\section{Goal}
The main goal of my research is to investigate the influence
of geometry and topology of the domain $\Omega$
on the solutions of the Cahn-Hilliard equation.

\section{The equation}

The Cahn-Hilliard equation:
%
\begin{equation}\label{eq:ch} \left\{ \begin{array}{ll}
u_t = \Delta(-\varepsilon^2 \Delta u + u^3 - u),
  & u : \Omega \times [0, \infty) \to \mathbb{R},\ \varepsilon > 0 \\
\frac{\partial u}{\partial n} = \frac{\partial \Delta u}{\partial n} = 0,
  & x \in \partial\Omega \\
u(x,0) = u_0(x),
  & x \in \Omega
\end{array} \right. \end{equation}
%
models phase separation of a binary mixture.
Some basic analytic properties of the solutions follow.

\subsection{Mass conservation}

Solutions of the equation conserve mass in the sense that:
\begin{eqnarray}{d\over dt}\int_\Omega u & = &
\int_\Omega u_t = \int_\Omega \Delta(-\varepsilon^2 \Delta u + u^3 - u)
=  \int_{\partial\Omega}\nabla\Big(-\varepsilon^2 \Delta u + u^3 - u\Big)\cdot\vec{n} \nonumber \\
& = & -\varepsilon^2 \int_{\partial\Omega} {\partial \Delta u \over \partial\vec{n}}
+ \int_{\partial\Omega} 3u^2{\partial u \over \partial\vec{n}}
- \int_{\partial\Omega} {\partial u \over \partial\vec{n}} = 0 \label{eq:mass}
\end{eqnarray}

\subsection{Steady states in $\Omega = \mathbb{R}$}
\label{sec:stationary}
For $\Omega = \mathbb{R}$,
function $f(x):= \tanh {x-x_0\over\varepsilon\sqrt{2}}$
is a steady state solution for the C-H equation.
Indeed: $-\varepsilon^2 f'' + f^3 - f = 0$.
It should be possible to show that this is a stable solution,
and that $f \equiv 0$ is an unstable solution.
What are the steady states when the domain is bounded?

For $\Omega = [0,L]$, every steady-state solution satisfies
the equation:
\begin{equation}
\label{eq:steady}
\left\{ \begin{array}{ll}
0 = -\varepsilon^2 u''''(x) + (u^3(x) - u(x))''  &  x \in [0,L] \\
0 = u'(x) = u'''(x)  &  x \in \{0,L\}
\end{array} \right. \end{equation}
If one sets the values $u(0)$, $u(L)$, $u''(0)$ and $u''(L)$,
does this determine a unique solution?
Does every choice of the above values yield a solution?

Consider the equation as in Equation \ref{eq:steady},
but let the domain be $\Omega = [0,+\infty)$, and boundary conditions:
\begin{equation}
u(0) = A, \ \ u'(0) = 0, \ \ u''(0) = B, \ \ u'''(0) = 0
\end{equation}
Under which conditions can the solution explode?
What is the relation between the solution of this problem
and the solutions of Equation \ref{eq:steady}?
How should this equation be solved numerically?

For numerical observations on the stability of very narrow domains,
see Section \ref{sec:narrow}.

\subsection{Scaling of $\varepsilon$}

Let $u^\ast(x,t)$ be the solution of Equation \ref{eq:ch}, and consider an equation:
\begin{equation}\label{eq:ch:lambda} \left\{ \begin{array}{ll}
u_t = \Delta(-(\lambda\varepsilon)^2 \Delta u + u^3 - u),
  & u : \Omega \times [0, \infty) \to \mathbb{R},\ \varepsilon > 0 \\
\frac{\partial u}{\partial n} = \frac{\partial \Delta u}{\partial n} = 0,
  & x \in \partial\Omega \\
u(x,0) = u_{0\lambda}(x),
  & x \in \Omega
\end{array} \right. \end{equation}
Furthermore, let $u_{0\lambda}(x) = u_0(\lambda x)$.
Then the solution of Equation \ref{eq:ch:lambda} is:
\begin{equation}
u^\ast_\lambda(x,t) = u^\ast(\lambda x, \lambda^2 t)
\end{equation}

\section{Solving the C-H equation numerically}

\subsection{Solving the equation using DCT}
I chose a scheme proposed by Vollmayr-Lee et al.~\cite{vollmayr03}
(here $\tilde{u}$ is the value from the previous time step):
\begin{eqnarray}
u & + & (1-a_1)\tau\Delta u
+(1-a_2)\varepsilon^2\tau\Delta^2 u
 -  (1-a_3) \tau \Delta (\tilde{u}^m u^{3-m}) \nonumber \\
& = & \tilde{u} - a_1 \tau\Delta \tilde{u} - a_2\varepsilon^2\tau\Delta^2 \tilde{u} + a_3\tau\Delta \tilde{u}^3
\end{eqnarray}
I set $a_2 = 0$, $a_3 = 1$ and $a_1 = a$.  The initial equation simplifies to:
\begin{eqnarray}
\label{eq:predct}
u + (1-a)\tau\Delta u +\varepsilon^2\tau\Delta^2 u
& = & \tilde{u} - a \tau\Delta \tilde{u} + \tau\Delta \tilde{u}^3
\end{eqnarray}

Due to the Neumann boundary conditions, it is possible
to solve the equation numerically using the Discrete Cosine Transform~\cite{nr}.
I use $N$ point lattice in 1D, and $N\times N$ point lattice in 2D.
% The DCT of $\nabla^2$ in 1D and 2D is, respectively:\footnote{I haven't really proved it yet, myself.}
% \begin{eqnarray}
% (\widehat{\Delta u})_l & = & \frac{1}{\xi^2} (2\cos\frac{\pi l}{N}-2)\widehat{u}_l \\
% (\widehat{\Delta u})_{lm} & = & \frac{1}{\xi^2} (2\cos\frac{\pi l}{N}+2\cos\frac{\pi m}{N} - 4)\widehat{u}_{lm}
% \end{eqnarray}
%
Applying DCT to the both sides of Equation~\ref{eq:predct} gives:
\begin{equation}
\label{eq:dct}
\widehat{u}
+ (1-a)\frac{\tau}{\xi^2} L \widehat{u}
+ \frac{\varepsilon^2\tau}{\xi^4} L^2 \widehat{u}
= \widehat{\tilde{u}}
+ \frac{\tau}{\xi^2} L \widehat{\tilde{u}^3}
- a \frac{\tau}{\xi^2} L \widehat{\tilde{u}}
\end{equation}
where $L$ denotes an unscaled Laplacian operator in the transformed space
(for details, see Appendix \ref{sec:dct}, Equations \ref{eq:L:1D}, \ref{eq:L:2D}):
\begin{eqnarray}
L(n) & = &  2\cos\frac{\pi n}{N}-2 \\
L(m,n) & = &  2\cos\frac{\pi m}{N}+2\cos\frac{\pi n}{N} - 4
\end{eqnarray}
Let's define:
\begin{equation}\lambda_1 = \frac{\tau}{\xi^2},\quad \lambda_2 = \frac{\varepsilon^2\tau}{\xi^4}
 = \frac{\varepsilon^2}{\xi^2}\lambda_1\end{equation}
\begin{equation}C = 1 + (1-a)\lambda_1 L + \lambda_2 L^2\end{equation}
Then Equation~\ref{eq:dct} can be rewritten as:
\begin{equation}\widehat{u} = \frac{\widehat{\tilde{u}} + \lambda_1 L (\widehat{\tilde{u}^3} - a \widehat{\tilde{u}})}{C}\end{equation}

Setting $a = 3$ should provide a stable scheme~\cite{vollmayr03}.
%I'm not 100\% sure whether this is independent of $\varepsilon$
%(the original scheme assumed $\varepsilon^2 = 1$).

The DCT implementation's time complexity is $O(N \log N)$ in 1D, and $O(N^2 \log N)$ in 2D.
The computations are dominated by DCT and its inverse.
The ``classic'' finite-difference method would have time complexity $O(N^2)$ and $O(N^4)$, respectively,
due to multiplication of a matrix,
of size $N\times N$ or $N^2\times N^2$ (analogous to the function $\frac{1}{C}$),
by a vector.



\subsection{Solving non-convex 2D domain cases using FEM}

The cases where the domain is more complex
are computed using finite element method.
The weak formulation of the problem is derived as follows:

\begin{equation} \left\{ \begin{array}{ll}
u_t & = \Delta w \\
w & = -\varepsilon^2 \Delta u + u^3 - u
\end{array} \right. \end{equation}

\begin{equation} \left\{ \begin{array}{ll}
\int_\Omega u_t \varphi - \int_\Omega \Delta w \varphi & = 0 \\
\int_\Omega w \psi + \varepsilon^2 \int_\Omega \Delta u \psi - \int_\Omega(u^3 - u) \psi & = 0
\end{array} \right. \end{equation}

\begin{equation} \left\{ \begin{array}{ll}
\int_\Omega u_t \varphi - \int_{\partial\Omega} (\nabla u \cdot \vec{n})\varphi + \int_\Omega \nabla w \nabla \varphi & = 0 \\
\int_\Omega w \psi + \varepsilon^2 \int_{\partial\Omega}(\nabla u \cdot \vec{n})\psi - \varepsilon^2 \int_\Omega \nabla u \nabla \psi - \int_\Omega(u^3 - u) \psi & = 0
\end{array} \right. \end{equation}

Boundary conditions imply that
$\nabla u \cdot \vec{n} = \frac{\partial u}{\partial \vec{n}} = 0$
and 
$\nabla w \cdot \vec{n} = \frac{\partial w}{\partial \vec{n}} = 0$
on $\partial\Omega$.
Therefore:

\begin{equation} \left\{ \begin{array}{ll}
\int_\Omega u_t \varphi + \int_\Omega \nabla w \nabla \varphi & = 0 \\
\int_\Omega w \psi - \varepsilon^2 \int_\Omega \nabla u \nabla \psi - \int_\Omega(u^3 - u) \psi & = 0
\end{array} \right. \end{equation}

Time discretization is done as in FDM case, by substituting $u$ with $a \tilde{u} + (1-a) u$
(where $\tilde{u}$ is the value from the previous time step):
\begin{equation} \left\{ \begin{array}{ll}
\int_\Omega \frac{u - \tilde{u}}{\tau} \varphi + \int_\Omega \nabla w \nabla \varphi & = 0 \\
\int_\Omega w \psi - \varepsilon^2 \int_\Omega \nabla u \nabla \psi - \int_\Omega(\tilde{u}^3 - a \tilde{u} - (1-a)u) \psi & = 0
\end{array} \right. \end{equation}




\subsection{Dealing with numerical artifacts}

Given the function $u$, it is interesting to calculate (for plotting, for example) some
functions depending on $u$.
Na\"ive calculation of some functions, such as:
\begin{itemize}
\item $\Delta(-\varepsilon^2 \Delta u + u^3 - u)$,
\item $-\varepsilon^2 \Delta u + u^3 - u$,
\item $-\varepsilon^2 {\partial \Delta u \over \partial {x}},
+ 3u^2{\partial u \over \partial {x}} - {\partial u \over \partial {x}}$ in 1D
\end{itemize}
produces numerical artifacts.
I think it is due to the fact that the equation is not solved by a fully implicit method.
In a way, the true function $u$ is represented by the pair: $u$ and $\tilde{u}$,
rather than by any of them.

The first two functions can be better approximated by:
\begin{itemize}
\item $\Delta \Big( -\varepsilon^2 \Delta u + \tilde{u}^3 - a \tilde{u} - (1-a)u \Big)$
\item $-\varepsilon^2 \Delta u + \tilde{u}^3 - a \tilde{u} - (1-a)u$
\end{itemize}

The third function is is more difficult, since an additional problem arises:
${\partial u \over \partial {x}}$ and ${\partial \Delta u \over \partial {x}}$
are centered at $x_i + \frac{\xi}{2}$, while $u$ is centered at $x$.
Centering the gradients at $x_i$ by approximating them with $\frac{v_{i+1} - v_{i-1}}{2 \xi}$
introduces artifacts too (probably because $u$ is steep, but $\nabla u$ is not constant).

Therefore, the idea is to use one of the following formulae (assuming domain $\Omega = [c, d]$):
\begin{itemize}
\item $\int_c^x \Big( -\varepsilon^2 \Delta^2 u + \Delta(\tilde{u}^3 - a \tilde{u} - (1-a)u) \Big)$
(centered at $x_i$)
\item $-\varepsilon^2 {\partial \Delta u \over \partial {x}}
+ {\partial \tilde{u}^3 \over \partial {x}}
- a {\partial \tilde{u} \over \partial {x}}
- (1-a) {\partial u \over \partial {x}}$
(centered at $x_i + \frac{\xi}{2}$)
\end{itemize}
which don't produce any visible artifacts and give similar shapes, but the values
are slightly scaled (by a factor of $2$, at most).
The $\int_c^x$ version gives the correct value at the right end
of the domain (and the same as the other version), so this is probably not a problem of summation.
I don't know which version is closer to the true value.

Currently, I don't know how to calculate: ${\Delta(-\varepsilon^2 \Delta u + u^3 - u) \over |\nabla u|}$.
Considering the previous experiences, I'm not sure if centering the gradient at $x_i$ is a good idea.

\subsection{Implementation}
The DCT version (both 1D and 2D) is implemented using MATLAB,
the FEM version is implemented using freeFEM++.
The implementation was done without good understanding of the methods.
The main purpose was to learn how the solutions of the equation might look like.

\subsection{Validation}
I have a fast, stable, mass-conserving FDM implementation in 1D and 2D,
using Vollmayr-Lee et. al. scheme~\cite{vollmayr03} and solving using DCT.
I can ``play with it'' and learn how the 1D solutions behave.
I could also try to reproduce the results from the literature.
Then I could move to 2D.  One possible
validation of 2D is to set $u_0(x,y) = u_0(x)$, that is, constant
along the second dimension, and check if the solutions are constant along
the second dimension over time.  Next, I could calibrate FEM implementation
so that it would reproduce the results of 2D FDM.
Finally, I could move to non-convex domains (the true benefit of FEM).






\section{Properties of the solutions}

\subsection{Mass transfer}

Equation \ref{eq:mass} states that mass, understood as $\int_\Omega u$ is constant in time.
Therefore, any change of mass in one part of the domain must be instantaneously countered
by a change in some other part.
I've started investigating the ``mass transfer'' in 1D, and in the future I'd like to
understand the process in 2D too.

In 1D, let $\Omega = [a,b]$, and consider
two ``islands'' of $1$ on two opposite ends of the domain,
and ``sea'' of $-1$ between them.
The smaller island will gradually disappear, and the larger island will {\it instantaneously}
gain the mass lost by the smaller one (total mass is conserved).
The larger the ``sea'', the longer the process will take.

The ``information'' about mass change is ``transmitted'' by
$\frac{\partial \Delta u}{\partial \vec{n}}$
and $\frac{\partial u}{\partial \vec{n}}$.
For $\Omega = [a,b]$, from Equation \ref{eq:mass} it follows that the function:
\begin{equation}
M(x) = -\varepsilon^2 {\partial \Delta u \over \partial {x}}
+ (3u^2 - 1){\partial u \over \partial {x}}
\end{equation}
gives the ``mass flux'', or the rate of change of mass in the subdomain $[a,x]$.

It could be interesting to control the mass transfer somehow.
The transfer could be controlled by setting an artificial
boundary at the point where $\frac{\partial u}{\partial x} = 0$
and setting appropriate BC for $\frac{\partial \Delta u}{\partial x}$.
Please note that the point where $\frac{\partial u}{\partial x} = 0$
is moving in time.

In 2D, it might be interesting to look closer at the curves given by
$u(x) = \pm {\sqrt{3} \over 3}$, since by taking this level set,
the mass flux simplifies to $-\varepsilon^2 {\partial \Delta u \over \partial \vec{n}}$.


\subsection{Interface motion}

After the initial stage, when domains are already formed,
the shape and motion of interfaces between the domains plays
a major role.

The interface speed can be defined as a speed at which a curve (in 2D case) defined by $x : u(x) = 0$ moves
in the direction of $\nabla u$.  This translates to:
\begin{equation}
v = \frac{{\partial u \over \partial t}}{{\partial u \over \partial \vec{g}}}|_{\vec{g} = \frac{\nabla u}{|\nabla u|}}
  = {\Delta(-\varepsilon^2 \Delta u + u^3 - u) \over \nabla u \cdot \frac{\nabla u}{|\nabla u|}}
  = {\Delta(-\varepsilon^2 \Delta u + u^3 - u) \over |\nabla u|}
\end{equation}

In 2D, how does the evolution of the interface curves look like?

In 1D, the speed of points $x$ for which $u(x) = 0$ seems to be inversely
proportional to the distances to other such points.


\subsection{Narrow domains}
\label{sec:narrow}

In 1D, when the domain is wide enough, a function similar in shape
to $\tanh {x-x_0\over\varepsilon\sqrt{2}}$ is a stable steady state,
and $f \equiv 0$ is an unstable steady state.

However, for sufficiently narrow domains, $f \equiv 0$ seems to
be the only steady state of the system (and it is stable).
The threshold width $w^\ast$ seems to be somewhere in the range
$2\varepsilon \leq w^\ast \leq 4\varepsilon$.
This property is extremely interesting in the context of building
complex, network-like domains in 2D: it may turn out that phase separation
will not occur in sufficiently narrow and curved channels!


%  \section{Solving 1D case using FDM}
%  %
%  In order to understand the behaviour of the Cahn-Hilliard equation,
%  a one-dimensional case is investigated first.
%  
%  The 1D case is solved numerically using finite differences method.
%  More specifically, a scheme proposed by de~Mello~et~al.~\cite{deMello05}
%  is implemented.
%  
%  $$\varepsilon^2 = 0.0001,\ dx = 0.01,\ dt = 0.0001,\ \Omega = [0, 1]$$
%  $$x_1 = 0,\ x_2 = dx,\ \dots,\ x_n = 1$$
%  $$\frac{u_i^{n+1}-u_i^n}{dt} =
%    -\varepsilon^2\Delta^2 u_i^{n+1}
%    + \Delta (u_i^n)^2u_i^{n+1}
%    - \Delta u_i^n$$
%  
%  % $$u_i^{n+1}
%  %   + dt \varepsilon^2 \Delta^2 u_i^{n+1}
%  %   - dt \Delta (u_i^n)^2u_i^{n+1}
%  %   =  u_i^n - dt \Delta u_i^n$$
%  
%  \begin{eqnarray*} u_i^{n+1} & + & \frac{dt \varepsilon^2}{dx^4}(
%        u_{i-2}^{n+1} - 4u_{i-1}^{n+1}
%        + 6 u_i^{n+1} - 4u_{i+1}^{n+1} + u_{i+2}^{n+1}) \nonumber \\
%    & - & \frac{dt}{dx^2}((u_{i-1}^n)^2u_{i-1}^{n+1}
%        - 2(u_i^n)^2 u_i^{n+1} + (u_{i+1}^n)^2 u_{i+1}^{n+1}) \nonumber \\
%    & = & u_i^n - \frac{dt}{dx^2}(u_{i-1}^n - 2u_i^n + u_{i+1}^n)
%  \end{eqnarray*}
%  
%  The boundary conditions are reflected by:
%  $$ u_0 = u_2,\ u_{-1} = u_3,\ u_{n+1} = u_{n-1},\ u_{n+2} = u_{n-2}$$
%  
%  In order to test the implementation,
%  numerical results by Furihata~\cite{furihata01} were taken as a reference.
%  The general behaviour was successfully reproduced,
%  although transitions occured at different timesteps,
%  and mass was not conserved.



\section{Current questions and problems}
\begin{itemize}
%  \item The time step in FDM was much too large, Rogers et al.~\cite{rogers88} suggest
%  $\Delta t \sim (\Delta x)^4$.  As far as I understand, this was suggested
%  for 2D lattice, but probably holds true for 1D too.
%  \item Papers propose various methods of dealing with the nonlinearity of $u^3 - u$:
%  \begin{itemize}
%  \item[$\clubsuit$] De Mello et al.~\cite{deMello05} propose: $u_{n+1} u_n^2 - u_n$.
%  \item[$\clubsuit$] Furihata~\cite{furihata01} proposes: ${(u_{n+1}^3 + u_{n+1}^2 u_n + u_{n+1} u_n^2 + u_n^3)\over 4} + {u_{n+1}+u_n \over 2}$.
%  \item[$\clubsuit$] Vollmayr et~al.~\cite{vollmayr03} propose:
%  $-(1-a_1)u_{n+1} - a_1 u_n + (1-a_3)u_{n+1}^{3-m} u_n^m + a_3u_n^3$
%  and show which $a_1$, $a_3$ and $m$ yield an unconditionally stable scheme.
%  \end{itemize}
\item I don't understand Eyre's theorem about {\it convexity splitting}~\cite{eyre98},
which is crucial for the scheme of Vollmayr-Lee et~at.~\cite{vollmayr03}.
Bertozzi et al.~\cite{bertozziXX} uses the scheme by Vollmayr-Lee et al. as well.

\item I need to find reasonable $\Delta x$ and $\Delta t$ for computations using the DCT.

% \item I'd like to apply the Vollmayr-Lee et~at.~\cite{vollmayr03}
% approach to the time discretization in FEM implementation.

\item Aggregation of two rounded islands into a larger object
is a common phenomenon in 2D.  I don't understand the ``mechanics''
or ``driving force'' of it.  For this reason I decided to implement
the 1D case and examine the aggregation process in a simpler case first.

\item In 1D, after the initial stage, it seems that the the state
of the system might be {\it almost completely} described by the values
$x_i$, $u'(x_i)$, $u''(x_i)$,
$u^{(3)}(x_i)$, $u^{(4)}(x_i)$ at the points where $u(x_i) = 0$.
I'd like to verify that.
Also, this idea needs a more precise formulation.

%  \item I would like to learn how to solve the Cahn-Hilliard equation
%  using Discrete Cosine Transform~\cite{nr}.
%  Also, I'd like to see if the Fourier transform, or DCT of $u$
%  contain any valuable information about the state of the system.
%  A MATLAB code downloaded from Eyre's website solves the C-H equation
%  using DCT.  I don't understand some details of the implementation.

\item I'd like to see the solutions of the 1D equation
\begin{equation}
0 = -\varepsilon^2 u''''(x) + (u^3)''(x) - u''(x)
\end{equation}
on $\Omega = [0,\infty)$ with boundary conditions
$u(0) = A$, $u'(0) = 0$, $u''(0) = B$, $u'''(0) = 0$.
I'd like to check how these solutions depend on $A$ and $B$.

\item It would be plausible to have an adaptive time discretization,
which would skim over the long periods when ``nothing happens''
and concentrate on the periods when transitions occur.
In order to do this, I would first have to understand what happens
when ``nothing happens'', and that's another reason to look carefully
at 1D solutions.  I'd also need a way to forecast the ``idle'' periods.

%The ``idle'' periods can be naturally defined as:
%$$\sup_\Omega |\Delta(-\varepsilon^2 \Delta u + u^3 - u))| < \delta$$
%but maybe there is a better criterion.

\item I'd like to see if the Fourier transform, or DCT of $u$
contain any valuable information about the state of the system.


\item I'd like to learn the stability analysis of stationary solutions
of evolution equations, and apply it to the analytic solution in 1D
(see Section \ref{sec:stationary}).  I'll try to learn it from \cite{cazenave}.

\item In 1D (and probably also in the higher dimensions),
in the later stages of development of the system,
the domain interfaces are locally ``similar'' to $\tanh{{x-x_0} \over \sqrt{2 \varepsilon^2}}$.
For instance, $u(x_0) = 0$ usually entails $u_x(x_0) = {1 \over \sqrt{2 \varepsilon^2}}$.
I don't know how to define this ``similarity'' precisely, and I have no idea how to convert
this observation into a theorem.


\item I cannot derive the fact that C-H process
is a gradient flow in the potential
$E(u) = \int_\Omega {1\over 2}\varepsilon^2 |\nabla u|^2 + {1\over 4}(u^2 - 1)^2$.
% All I can do is derive:
% $$E'(u)\varphi = \int_\Omega \Big(
% -\varepsilon^2 \Delta u + (u^3 - u) \Big)\varphi,
% \ \forall \varphi \in C^2(\Omega)$$
% which might be unrelated.
The derivation can be found, for instance, in \cite{fife00},
I just need to understand it.
% and I suspect that later I could use the equation:
% $$u_t + \nabla \cdot \mathbf{F}(u) = 0$$
% (see Chapter 11 of \cite{evans98:ams}).
% My problem probably has something to do with the
% fact that the gradient flow is in the $H^{-1}$ norm,
% which I don't know.
\end{itemize}



\appendix

\section{Discrete Cosine Transform} \label{sec:dct}

There is a number of ways in which Discrete Cosine Transform
can be defined.
Therefore, for the sake of clarity, I define the DCT and prove
some necessary properties.
The definition is, up to indices, equivalent to the DCT function
of MATLAB.  The 2D version is, up to indices, equivalent
to the DCT2 function of MATLAB.

\subsection{DCT in 1D}

A sequence $\{f_n\}_{n=0,\dots,N-1}$, where $f_n \in \mathbb{R}$,
has a DCT $\{F_k\}_{k=0,\dots,N-1}$ defined as:
\begin{equation}
F_k := \mathscr{C}[f](k) := w(k)\sum_{n=0}^{N-1} f_n \cos{\pi(n+\frac{1}{2})k \over N}
\end{equation}
where
\begin{equation}
\label{eq:def:w}
w(k) := \left\{ \begin{array}{ll}
\sqrt{1 \over N} & k = 0 \\
\sqrt{2 \over N} & k \neq 0
\end{array} \right.  
\end{equation}

The inverse transform is given by:
$$f_n := \mathscr{C}^{-1}[F](k) := {\sum_{k=0}^{N-1}} w(k) F_k \cos{\pi(n+\frac{1}{2})k \over N}$$

Indeed, $\mathscr{C}[\mathscr{C}^{-1}[F]] = F$ and $\mathscr{C}^{-1}[\mathscr{C}[f]] = f$:
\begin{eqnarray}
\mathscr{C}^{-1}[\mathscr{C}[f]](m) & = & {\sum_{k=0}^{N-1}} w(k)
\Big( w(k)\sum_{n=0}^{N-1} f_n \cos{\pi(n+\frac{1}{2})k \over N} \Big)
\cos{\pi(m+\frac{1}{2})k \over N} \nonumber \\
%
& = & \sum_{n=0}^{N-1} f_n \sum_{k=0}^{N-1}
\Big( w(k) \cos{\pi(n+\frac{1}{2})k \over N} \Big)
\Big( w(k) \cos{\pi(m+\frac{1}{2})k \over N} \Big) \nonumber \\
%
& = & \dots = \sum_{n=0}^{N-1} f_n \delta_{nm} = f_m \\
%
%
\mathscr{C}[\mathscr{C}^{-1}[F]](l) & = &
w(l)\sum_{n=0}^{N-1}
\sum_{k=0}^{N-1} w(k) F_k \cos{\pi(n+\frac{1}{2})k \over N}
\cos{\pi(n+\frac{1}{2})l \over N} \nonumber \\
%
& = & \sum_{k=0}^{N-1} F_k \sum_{n=0}^{N-1}
\Big( w(k) \cos{\pi(n+\frac{1}{2})k \over N} \Big)
\Big( w(l) \cos{\pi(n+\frac{1}{2})l \over N} \Big) \nonumber \\
%
& = & \ldots = \sum_{k=0}^{N-1} F_k \delta_{kl} = F_l
\end{eqnarray}

It is clear that both $\mathscr{C}$ and $\mathscr{C}^{-1}$ are linear.
Sometimes $\mathscr{C}[f]$ is written as $\widehat{f}$.
Assuming Neumann boundary conditions: $u_{-1} := u_{0}$ and $u_{N} := u_{N-1}$,
the cosine transform of Laplacian is:
\begin{eqnarray}
(\widehat{\Delta u})_k
& = & w(k)\sum_{n=0}^{N-1} (\Delta u)_n \cos{\pi(n+\frac{1}{2})k \over N} \nonumber \\
& = & w(k)\sum_{n=0}^{N-1} \frac{u_{n-1} - 2u_n + u_{n+1}}{\xi^2} \cos{\pi(n+\frac{1}{2})k \over N} \nonumber \\
& = & \frac{1}{\xi^2}\Big( w(k)\sum_{n=-1}^{N-2} u_{n} \cos{\pi(n+\frac{3}{2})k \over N}
- 2 \widehat{u}_k
+ w(k)\sum_{n=1}^{N} u_{n} \cos{\pi(n-\frac{1}{2})k \over N}\Big) \nonumber \\
%
& = & \frac{1}{\xi^2}\Big( w(k)\sum_{n=0}^{N-1} u_{n} \cos{\pi(n+\frac{3}{2})k \over N}
+ w(k) u_{-1} \cos{\pi(+\frac{1}{2})k \over N} - w(k) u_{N-1} \cos{\pi(N+\frac{1}{2})k \over N} \nonumber \\
& - & 2 \widehat{u}_k
+ w(k)\sum_{n=0}^{N-1} u_{n} \cos{\pi(n-\frac{1}{2})k \over N}
+ w(k) u_N \cos{\pi(N-\frac{1}{2})k \over N} - w(k) u_0 \cos{\pi(-\frac{1}{2})k \over N} \Big) \nonumber \\
%
& = & \frac{1}{\xi^2}\Big( w(k)\sum_{n=0}^{N-1} u_{n} \cos{\pi(n+\frac{3}{2})k \over N}
- 2 \widehat{u}_k
+ w(k)\sum_{n=0}^{N-1} u_{n} \cos{\pi(n-\frac{1}{2})k \over N} \nonumber \\
& + &  w(k) \cos{\pi\frac{1}{2}k \over N}(u_{-1} - u_0 + u_N - u_{N-1}) \Big) \nonumber \\
%
& = & \frac{1}{\xi^2}\Big( w(k)\sum_{n=0}^{N-1} u_{n} (\cos{\pi (n+\frac{1}{2}) k \over N} \cos{\pi k \over N} - \sin{\pi (n+\frac{1}{2}) k \over N} \sin{\pi k \over N})
- 2 \widehat{u}_k \nonumber \\
& + & w(k)\sum_{n=0}^{N-1} u_{n} (\cos{\pi (n+\frac{1}{2}) k \over N} \cos{\pi k \over N} + \sin{\pi (n+\frac{1}{2}) k \over N} \sin{\pi k \over N}) \Big)
\nonumber \\
& = & \frac{1}{\xi^2}\Big( 2 \cos{\pi k \over N} w(k)\sum_{n=0}^{N-1} u_{n} \cos{\pi (n+\frac{1}{2}) k \over N} - 2 \widehat{u}_k \Big) \nonumber \\
& = & \frac{1}{\xi^2} (2 \cos{\pi k \over N} - 2) \widehat{u}_k
\end{eqnarray}

Let's define function $L(k)$ representing an unscaled Laplacian in the transformed space:
\begin{equation}
\label{eq:L:1D}
L(k) := 2 \cos{\pi k \over N} - 2
\end{equation}

\subsection{DCT in 2D}
The two-dimensional DCT of a matrix $(f_{mn})_{m=0,\dots,M-1;n=0,\dots,N-1}$,
where $f_{mn} \in \mathbb{R}$,
is a matrix $(F_{kl})_{k=0,\dots,M-1;l=0,\dots,N-1}$ defined as:
\begin{equation}
F_{kl} := \mathscr{C}[f](k,l) := w(k)w(l)
\sum_{m=0}^{M-1} \sum_{n=0}^{N-1}
f_{mn} \cos{\pi(m+\frac{1}{2})k \over M} \cos{\pi(n+\frac{1}{2})l \over N}
\end{equation}
where $w$ is defined as in Equation \ref{eq:def:w}.
For now, the following facts about the 2D DCT are left without proof.

The result of DCT in two dimensions is equal to the result of DCT in the first dimension
applied to the DCT in the second dimension, and the order of application is irrelevant:
\begin{eqnarray}
\mathscr{C}_{xy}[f] := \mathscr{C}_x[\mathscr{C}_y[f]] &=& \mathscr{C}_y[\mathscr{C}_x[f]]
\end{eqnarray}
The 2D version of Equation \ref{eq:L:1D} is:
\begin{eqnarray}
\label{eq:L:2D}
\mathscr{C}_{xy}[\Delta f](k,l) &=& (2 \cos{\pi k \over N} + 2 \cos{\pi l \over N} - 4) \mathscr{C}_{xy}[f](k,l)
\end{eqnarray}


\nocite{bertozziXX}
\nocite{vollmayr03}
\nocite{deMello05}
\nocite{eyre98}
\nocite{nr}
\nocite{cazenave}

\nocite{wei98}
\nocite{rogers88}
\nocite{guillot04}
\nocite{guillot02}
\nocite{bloemker01}
\nocite{bates00}
% \nocite{sial03}

\nocite{grant93}
\nocite{fife00}

\bibliography{cahn-hilliard}


\end{document}
